---
title: "toronto vs. IL"
author: "SB"
date: "Wednesday, February 04, 2015"
output: pdf_document
---

# graphing IL from Toronto example

## may 7, 2017

modify the toronto example to use the 
Israeli data
## load setup code, simulate thesis state 
```{r setup}
library(RColorBrewer)
library(maptools)
library(ggmap)
library(rgeos)
library(censusFunctions) 
importData("savedGoogleMap")
# Read the neighborhood shapefile data and plot
setwd("~/scratch/toronto_neighbourhoods")
shpfile <- "NEIGHBORHOODS_WGS84_2.shp"
sh <- rgdal::readOGR(shpfile)
plot(sh)

# Add demographic data
# The neighbourhood ID is a string - change it to a integer
sh@data$AREA_S_CD <- as.numeric(sh@data$AREA_S_CD)

# Read in the demographic data and merge on Neighbourhood Id
demo <- read.csv(file="WB-Demographics.csv", header=T)
sh2 <- merge(sh, demo, by.x='AREA_S_CD', by.y='Neighbourhood.Id')


```
now, what does this col=cols line do?


```{r second}


# Set the palette
# creates 128 shades between white and red
p <- colorRampPalette(c("white", "red"))(128)
palette(p)

# Scale the total population to the palette
pop <- sh2@data$Total.Population
cols <- (pop - min(pop))/diff(range(pop))*127+1
# what is this cols variable?
# View(cols) #don't call view in ess, only Rstudio'
pop  # city-section populations range from 7k to 50k
cols[1]
range(cols)
pop[12]
# what column names are there?
names(sh2@data)
```
What is the population of the 12th area?
```{r pop1}
pop[12]
```
What is the name of that area?
```{r nme}
sh2@data$AREA_NAME[12]
```
Does this area have any aliases?
```{r rco}
sh2@data$Neighbourhood[12]
```
### Can you say any of that with inline code?
The population of region `sh2@data$AREA_NAME[12]` aka
`sh2@data$Neighbourhood[12]` was `pop[12]` in 1998.
By 2010 it had grown to `pop[22]`.
This paragraph is interesting because, blah.

```{r more}
plot(sh, col = cols)

#RColorBrewer, spectral
p <- colorRampPalette(brewer.pal(11, 'Spectral'))(128)
palette(rev(p))
plot(sh2, col = cols)

#GGPLOT 
points <- fortify(sh, region = 'AREA_S_CD')

# Plot the neighborhoods
toronto <- qmap("Toronto, Ontario", zoom = 10)
toronto + geom_polygon(aes(x=long,y=lat, group=group, alpha=0.25), data=points, fill='white') + geom_polygon(aes(x=long,y=lat, group=group), data=points, color='black', fill=NA)

# merge the shapefile data with the social housing data, using the neighborhood ID
points2 <- merge(points, demo, by.x='id', by.y='Neighbourhood.Id', all.x=TRUE)

# Plot
toronto + geom_polygon(aes(x=long,y=lat, group=group, fill=Total.Population), data=points2, color='black') + 
  scale_fill_gradient(low='white', high='red')

# Spectral plot
toronto + geom_polygon(aes(x=long,y=lat, group=group, fill=Total.Population), data=points2, color='black') + 
  scale_fill_distiller(palette='Spectral') + scale_alpha(range=c(0.5,0.5))
```
